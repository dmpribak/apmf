\documentclass[mathserif]{beamer}
\usetheme{metropolis}
\metroset{block=fill}

\usepackage{amsmath,amssymb}
\usepackage{tabularray}
\usepackage{xcolor}
\usepackage{graphicx}
\usepackage{algpseudocode}
\usepackage{tikz}
\usetikzlibrary{graphs}
\usetikzlibrary{graphs.standard}
\usetikzlibrary{quotes}

\begin{document}
\title{All-Pairs Max-Flow Algorithms and Implementations}
\author{Dillan Pribak}


\begin{frame}
\maketitle
\end{frame}

\begin{frame}
	\frametitle{Naive Approach}
	Run a max flow algorithm on all vertex pairs.\\
	Let $G(V,E,c)$ be a directed graph with integral capacities.\\
	$m = |E|$\\ 
	$n = |V|$\\
	$f=$ max flow in G\\
	\textbf{All-Pairs Ford-Fulkerson:} $O(fm\cdot n^2)= O(fmn^2)$\\
	\textbf{All-Pairs Dinitz:} $O(mn^2\cdot n^2) = O(mn^4)$
	
\end{frame}

\begin{frame}
	\frametitle{Implementation Details: Overview}
	\begin{block}{My work:}
		\begin{itemize}
			\item Functions for building and updating residuals.
			\item DFS, BFS, blocking flow algorithms.
		\end{itemize}
	\end{block}
	Written in Python.\\
	Used \texttt{networkx} package for graph data structure.\\
	Implementations are not fully optimized.

	
\end{frame}

\begin{frame}
	\frametitle{Implementation Details: Ford-Fulkerson}

	\begin{algorithmic}[1]
		\Procedure{Ford-Fulkerson}{$s,t,G=(V,E)$}
			\State $f \gets 0$	
			\State $G_{f} \gets G$
			\While{ there is a path $P$ from $s$ to $t$ in $G_{f}$}
				\State $f' \gets$ maximum flow along $P$
				\State Update residual $G_{f}$ accordingly
				\State $f \gets f + f'$ 
	\end{algorithmic}
	\begin{block}{Custom DFS:}
		\begin{itemize}
			\item Returns list of edges in path from $s$ to $t$.\\
			\item Ignores zero-weight edges.\\
			\item Return path if $t$ found, otherwise return ``no path found''.
		\end{itemize}
	\end{block}
\end{frame}

\begin{frame}[fragile]
	\frametitle{Implementation Details: Dinitz}
		\begin{algorithmic}[1]	
			\Procedure{Dinitz}{$s,t,G=(V,E)$}
				\State $f \gets 0$
				\While{$f$ is not a max flow}	
					\State Let $G_{f}$ be the residual
					\State Let $E'$ be edges in all shortest paths from $s$ to $t$
					\State $f' \gets$ any blocking flow in $G'=(V,E')$
					\State $f \gets f + f'$
		\end{algorithmic}
	\begin{block}{Blocking Flow}
		\begin{itemize}
			\item Find path from $s$ to $t$ (DFS).
			\item Push maximum flow through that path.
			\item Continue until no paths to $t$.
		\end{itemize}	
	\end{block}

\end{frame}

\begin{frame}
	\frametitle{Implementation Details: Dinitz}
	\begin{block}{Custom BFS:}
		\begin{itemize}
			\item Each iteration expands search out by one.
			\item Once $t$ is found, finish that level to check for other paths.
			\item Maintains a parent list for backtracking where vertices can have multiple parents.
			\item Returns all edges in shortest paths from $s$ to $t$.
		\end{itemize}
	\end{block}
	\begin{figure}[h]
		\centering	
		\tikz \graph [nodes={draw,circle}] {
			s -> {a,b} -> c -> t
		};
		\caption{Node $c$ has two parents in all shortest paths.}
		\label{fig:}
	\end{figure}
\end{frame}

\begin{frame}
	\frametitle{Better Approach: Gomory-Hu Tree}
	\begin{definition}
		A \alert{Gomory-Hu Tree} is a tree on the vertices of $G$ where the minimum edge weight in the path between two vertices is the max-flow between those vertices.
	\end{definition}

	\textbf{Interpretation:} A tree where the edges are all the bottlenecks.
	\begin{figure}[h]
		\centering
		\tikz \graph[nodes={draw, circle}, no placement]{
			{a[at={(0:0)}]}--{b[at={(1,1)}],c[at={(1,-1)}]},b--c,c--d[at={(2,0)}],a--d,b--d,d--e[at={(3,0)}],e--{f[at={(4,1)}],g[at={(4,-1)}]},f--g
		};
		\caption{Passage from left side to right side bottlenecked by $(e,f)$ edge.}
		\label{fig:}
	\end{figure}	
\end{frame}

\begin{frame}
	\frametitle{Better Approach: Gomory-Hu Tree}
	\textbf{Example:}
	
	\begin{figure}[h]
		\centering
	\tikz \graph[nodes={draw,circle}, no placement] {
		a[at={(0:0)}],b[at={(1,0)}],c[at={(2,1)}],d[at={(3,0)}],e[at={(2,-1)}],
		a--["1"]b,a--["1"']e,b--["1"]{c,e},c--{b,d},d--["1"']c,d--["1"]e,e--{a,b}
	};
	\hspace{20pt}
	\tikz \graph[nodes={draw,circle}, no placement] {
		a[at={(0:0)}],b[at={(1,0)}],c[at={(2,1)}],d[at={(3,0)}],e[at={(2,-1)}],
		a--["2"]b,b--["2"]c,b--["3"]e,c--["2"]d
	};

		\caption{A flow network (left) and its Gomory-Hu tree (right)}
		\label{fig:g}
	\end{figure}

\end{frame}

\begin{frame}
	\frametitle{Finding Gomory-Hu Tree}
	\begin{block}{Algorithm [Gomory, Hu 1961]}
		Start with all vertices in a big pot. Split into smaller pots via min-cuts. Continue until all pots have one node.	
	\end{block}

	\begin{figure}[h]
	\centering
	\tikz \graph[nodes={draw,circle}, no placement]{
		abcdef
	};	
	\hspace{20pt}
	\tikz \graph[nodes={draw,circle}, no placement]{
		abe[at={(0:0)}],cd[at={(1.5,0)}],
		abe--["2"]cd
	};
	\hspace{20pt}
	\tikz \graph[nodes={draw,circle}, no placement]{
		a[at={(0:0)}],be[at={(1.5,0)}],cd[at={(3,0)}],
		a--["2"]be--["2"]cd
	};

	\vspace{10pt}
	\tikz \graph[nodes={draw,circle}, no placement]{
		a[at={(0:0)}],b[at={(1,0)}],e[at={(1,-1)}],cd[at={(2.5,0)}],
		a--["2"]b--["2"]cd,b--["3"]e
	};
	\hspace{20pt}
	\tikz \graph[nodes={draw,circle}, no placement]{
		a[at={(0:0)}],b[at={(1,0)}],e[at={(1,-1)}],c[at={(2,0)}],d[at={(2,-1)}],
		a--["2"]b--["2"]c,b--["3"]e,c--["2"]d
	};
	\caption{Progression of Gomory-Hu algorithm on graph from figure \ref{fig:g}}
	\end{figure}
\end{frame}

\begin{frame}
	\frametitle{Gomory-Hu Algorithm Analysis}
		\begin{block}{Generating Tree}		
		\begin{itemize}
			\item Creates $(n-1)$ splits $\implies$ makes $(n-1)$ calls to max-flow.\\
			\item Better than naive method! (Makes $O(n^2)$ calls to max-flow).\\
		\end{itemize}
		\end{block}

		\begin{block}{Accessing Tree}
			\begin{itemize}
		\item A flow value can be retrieved efficiently using a shortest path algorithm.\\
		\item To get constant access time, convert into a table in $O(n^2)$ time using a good APSP algorithm.
		\end{itemize}
		\end{block}

		\textbf{Note:} Only works on undirected graphs!
\end{frame}

\begin{frame}
	\frametitle{Implementation Details: Gomory-Hu Algorithm}
	
\end{frame}


\end{document}
